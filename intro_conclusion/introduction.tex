
\glsadd{q}
\glsadd{ho}
\glsadd{hamiltonian}

\Gls{qm} is the study of nature's fundamental processes, manifest in its smallest particles. 
\gls{qm} has been at the forefront of physics since the early $20^{\textrm{th}}$ century \cite{jammer1966conceptual}.
Advances in theoretical understanding of quantum mechanical systems in the first half of the century 
    \cite{einstein1905heuristic, born1926quantenmechanik, schrodinger1926undulatory, heisenberg1985quantentheoretische, von2018mathematical}
    came to underpin many modern information and communications technologies\footnotemark \ in the second half
    \cite{svelto2010principles, van2004principles}.
The $21^{\textrm{st}}$ century, on the other hand, is poised to see the development of 
    technologies which \emph{deliberately} exploit the most intricate quantum processes \cite{feynman1982simulating}. 
These \emph{quantum technologies} share the promsie of super-classical outcomes,
    i.e. that exploiting quantum phenomena can yield results which could never 
    be achieved through non-quantum (or \emph{classical}) means.  
The enduring motivation for the development of quantum technologies is \emph{quantum simulation}:
    controlling quantum systems to represent other quantum systems, 
    enabling the study of such structures and interactions for the first time \cite{feynman1982simulating, lloyd1996universal}. 
\par 

\footnotetext{Colloquially referred to as \emph{Quantum 1.0}.}
\par 

Significant advances in the design and construction of quantum hardware in 
    recent years promise reliable, large scale quantum infrastructure
    in the near future \cite{arute2019quantum, zhong2020quantum}. 
Alongside improvements in quantum systems' control, progress in quantum algorithms and software 
    foreshadow impactful applications for quantum technologies, 
    from database search \cite{grover1997quantum} to quantum chemistry \cite{cao2019quantum} and drug design \cite{cao2018potential}.
Automated methodologies for characterising quantum systems are among the applications becoming feasible, 
    with the development of quantum devices capable of
    simulating nature at the quantum level \cite{childs2018toward}. 
There is a large and growing interest in automatically identifying the \emph{models} of quantum systems, 
    i.e. the mathematical structure representing a system's interactions
    \cite{rigo2020machine, cranmer2020discovering, bairey2019learning, pickard2011ab, chertkov2018computational}.
\par 

In parallel to the rise of quantum technologies over the past several decades,
    \gls{ml} and \acrlong{ai} have enjoyed increasing interest and resources.
Landmark outcomes, for example in facial recognition \cite{taigman2014deepface} 
    and complex strategy games \cite{silver2017mastering, brown2019superhuman},
    were bolstered by dramatic gains in the design of information processing machinery
    such as supercomputing facilities \cite{top500} and \acrlongpl{gpu} \cite{lindholm2008nvidia}.
\Gls{ml} has been widely adopted to accelerate the impact of quantum technologies, 
        from error correction \cite{chen2019machine, valenti2019hamiltonian}
        to metrology \cite{hentschel2010machine}
        and device calibration \cite{lennon2019efficiently}.
    
\par 

In this thesis we report progress in the domain of quantum system characterisation,
    through novel quantum algorithms empowered by the promise of quantum simulators, 
    leveraging state-of-the-art machine learning techniques. 
Namely, we introduce and develop the \gls{qmla} as a powerful platform for the study of 
    quantum systems, ranging from controlled quantum simulators to experimental setups.
\gls{qmla} distills an approximate model for a given quantum system, 
    by constructing a series of candidate models and testing them against data from the system of interest.
In providing a robust software framework for \gls{qmla}, we initiate an exciting field of research 
    at the overlap of \acrlong{ml} and quantum simulation, 
    with proposed applications in calibrating new quantum technologies as well as understanding 
    quantum processes in nature.

\section{Thesis outline}

The works presented in this thesis are closely related, 
    all stemming from the \gls{qmla} protocol and framework. 
The thesis is organised into \emph{parts}, 
    which group together related bodies of work;
    within each part, individual studies are presented in self contained chapters.
The main results and novel research is contained in \crefrange{part:algorithms}{part:experimental_study}.  
At the outset of each part, we summarise its chapters and contributions.
The contents of each part are as follows. 

\begin{description}
    \item[\cref{part:contextual_review} Contextual Review] \
    
        We introduce the concepts upon which the thesis will build. 
        \cref{chapter:qm} establishes the vocabulary of \acrlong{qm}, 
            followed by a summary of \acrlong{ml} in \cref{chapter:ml}. 
        In both cases, we seek to introduce the minimal nomenclature required to 
            contextualise the work in the following chapters;
            that is, neither topic is described exhaustively. 
    
    \item[\cref{part:algorithms} Algorithms] \
    
        We provide a thorough explanation of the algorithms underlying this thesis. 
        We start by summarising \acrlong{qhl}, which serves as a key subroutine within later studies and should therefore 
            be understood, in order that the contributions of later chapters may be fully appreciated.
        The first major result is the \gls{qmla} algorithm itself, 
            detailed in \cref{chapter:qmla}. 
        All subsequent chapters assume knowledge of the terminology and concepts related to \gls{qmla}, 
            so unfamiliar readers will find \cref{chapter:qmla} essential. 
        Our next contribution is an open source software platform for the implementation of \gls{qmla}, 
            for the study of arbitary quantum systems.
        In \cref{chapter:sw} we list the implementation details of this framework, 
            but do not further any physical or algorithmic concepts. 
        The \gls{qmla} software is available at \cite{flynn2021QMLA} with documentation at \cite{qmla_docs}.
        By the end of \cref{part:algorithms}, 
            we are armed with \gls{qmla} as a tool for the inspection of target quantum systems of interest,
            which will serve as a platform for the remaining chapters.
        
    
    \item[\cref{part:theoretical_study} Theoretical Study] \
    
        We perform tests of the \gls{qmla} framework under idealised simulated conditions, 
            corresponding directly to \cite{flynn2021Quantum}. 
        In \cref{chapter:lattices} we demonstrate that \gls{qmla} is trustworthy in the most straightforward scenario, 
            where a number of candidate models are proposed in advance. 
        We then move to more difficult conditions in \cref{chapter:ga}, 
            exploring spaces of $10^5$ valid candidate models, by incorporating a \acrlong{ga} within \gls{qmla}. 
        Together, the cases studied here verify that \gls{qmla} shows promise in characterising quantum systems, 
            in particular suggesting a compelling application in the calibration and verification of quantum simulators. 
    
    \item[\cref{part:experimental_study} Experimental Study] \

        The final contribution reflects work published in \cite{gentile2020learning}.
        We extend the \gls{qmla} protocol to \emph{realistic} quantum systems, namely targeting the decoherence processes dominating the 
        dynamics of an electron spin in a \acrlong{nvc}. 
        In \cref{chapter:nv} we operate on data extracted from an experimental system, 
            from which \gls{qmla} distills models with high predictive power -- i.e. which can reproduce the dynamics of the target system -- 
            and which are in agreement with theoretical predictions. 
        \cref{chapter:nv} relied on several constraints to facilitate the model search; 
            in \cref{chapter:many_qubits} we relax some of those constraints by simulating a similar system, 
            and again exploiting a \acrlong{ga} to explore the model space. 
        The results of this part indicate that \gls{qmla} may be helpful in the study of \emph{black-box} quantum systems.
    
    \item[\cref{part:conclusion} Conclusion] \
    
        We close the thesis with a brief summary of its main contributions, 
            and offer an outlook for model learning methodologies in the context of quantum technologies. 

\end{description}
