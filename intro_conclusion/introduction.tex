\glsresetall
Significant advances in the design and construction of quantum hardware in 
    recent years promise reliable, large scale quantum infrastructure
    in the near future \cite{arute2019quantum, zhong2020quantum}. 
Alongside improvements in quantum systems' control, progress in quantum algorithms and software 
    foreshadow impactful applications for quantum technologies, 
    from database search \cite{grover1997quantum} to quantum chemistry \cite{cao2019quantum} and drug design \cite{cao2018potential}.
Automated methodologies for characterising quantum systems are among the applications becoming feasible, 
    with the development of quantum devices capable of
    simulating nature at the quantum level \cite{childs2018toward, arute2019quantum}. 
There is a large and growing interest in automatically identifying the \emph{models} of quantum systems, 
    i.e. the mathematical structure representing the system's interactions
    \cite{rigo2020machine, cranmer2020discovering, bairey2019learning, pickard2011ab, chertkov2018computational}.
\par 

In parallel to the rise of quantum technologies over the past several decades,
    \glsentrylong{ml} and \glsentrylong{ai} have enjoyed increasing interest and resources.
Landmark outcomes, for example in facial recognition \cite{taigman2014deepface} 
    and complex strategy games \cite{silver2017mastering, brown2019superhuman},
    were bolstered by dramatic gains in the design of information processing devices
    such as supercomputing facilities and \glsentrylongpl{gpu} \cite{top500, lindholm2008nvidia}.

\par 

In this thesis we report progress in the domain of quantum system characterisation,
    through novel quantum algorithms empowered by the promise of quantum simulators, 
    leveraging state-of-the-art machine learning techniques. 
Namely, we introduce and develop the \gls{qmla} as a powerful platform for the study of 
    quantum systems ranging from controlled quantum simulators to experimental setups.
\gls{qmla} uncovers an approximate model for a given quantum system, 
    by constructing a series of candidate models and testing them against data from the system of interest.
In providing a robust software framework for \gls{qmla}, we initiate an exciting field of research 
    at the overlap of \gls{ml} and quantum simulation, 
    with proposed applications in calibrating new quantum technologies as well as understanding 
    quantum processes in nature.

\section{Main Results}

\section{Outline}

\section{Publications} % maybe as unnumberred chapter?