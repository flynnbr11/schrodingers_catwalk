\glsresetall
% TODO extend conclusion - at least two pages!

Optimal control techniques are a crucial component in improving quantum technologies, 
    such that imperfect near-term devices may be leveraged to achieve some meaningful quantum advantages. 
The developments presented in this thesis contribute to the growing interest in automatic characterisation
    and verification of quantum systems and devices. 
Namely, the introduction of the \gls{qmla} represents an important advancement, 
    whereby quantum systems can be completely characterised starting with little prior knowledge. 
The majority of this thesis was dedicated to the rigourous testing of \gls{qmla},
    gradually moving from ideal scenarios in simulation to genuine experimental quantum systems.
\par

We described the implementation of \gls{qmla} as an open source software platform in \cref{part:algorithms}, 
    detailing numerous tunable aspects of the protocol, and their impact on training candidate models in \cref{chapter:qhl}.
\gls{qmla} facilitates customisation of its core elements and subroutines, 
    such that it is applicable to a wide range of target quantum systems, as described in \crefrange{chapter:qmla}{chapter:sw}.
This malleability enables users to easily adapt the framework to their own needs,
    and formed the basis for the cases studied in the remainder of the thesis:
    we tested \gls{qmla} by devising a series of exploration strategies,
    each corresponding to a different target quantum system.

\par 
In \cref{part:theoretical_study} we considered ideal theoretical quantum systems in simulation. 
Initial tests in \cref{chapter:lattices} showed that \gls{qmla} could distinguish between 
    different physical scenarios and internal configurations. 
In \cref{chapter:ga}, we explored much larger model spaces by incorporating a \gls{ga} into \gls{qmla}'s model design;
    the \gls{ga} showed promise for characterising complex quantum systems by successfully identifying the target model.
The performance of the \gls{ga}, however, came at the expense of relying on a restrictive subroutine 
    -- used for training individual candidate models -- 
    drastically reducing its applicability to realistic systems. 
However, the restriction is permitted in the scope of characterising \emph{controlled} quantum systems, 
    for example new, untrusted quantum simulators. 

\par 

We concluded the thesis by considering realistic quantum systems in \cref{part:experimental_study}. 
Experimental data from an electron spin in a \gls{nvc} was treated in \cref{chapter:nv};
    this too relied upon tailoring \gls{qmla}'s procedure with respect to the system under study.
A theoretically justified Hamiltonian is proposed by \gls{qmla} to describe the decoherence of the electron spin, 
    yielding a highly predictive model in agreement with the system's measured dynamics, 
    albeit exploring a small model space. 
To overcome concerns that the model search was artifically constrained in the context of realistic systems, 
    \cref{chapter:many_qubits} exercised \gls{qmla} in a vast model space, 
    spanning terms which represent plausible interactions for the same type of system. 
Here, again, \gls{qmla} achieved high success rates, but with caveats on the subroutines assumed for model training, 
    and resorting to simulated data. 
\par

In summary, this thesis has provided extensive tests of the \gls{qmla} algorithm, 
    but each may be undermined by its individual constraints. 
In outlook, near-term developments of model learning methodologies in the context of quantum systems
    must address these shortcomings, for instance by unifying the strategies described in this thesis. 
Further, we anticipate immediate application in the study of open quantum systems, 
    by replacing the Hamiltoinian formalism examined here with a Lindbladian representation, 
    permitted within the \gls{qmla} apparatus.    
Through the advancements presented herein, we hope to have provided a solid foundation upon which these constraints may be relaxed, 
    ultimately with a view to providing an automated platform for the complete characterisation of quantum systems.
We envision \gls{qmla} as a straighforward but powerful utility for quantum engineers in the design of near term quantum devices, 
    expecting continued development of the framework alongside the burgeoning open-source quantum software eco-system. 
\par


