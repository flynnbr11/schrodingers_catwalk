\chapter*{Acknowledgements}
\addcontentsline{toc}{chapter}{Acknowledgements}

% Reflection
I was never one of those people destined for a PhD.
I don't think I knew what a PhD was until half way through my undergraduate studies, 
    by which time I was resigned to being tapped on the shoulder one day and asked to leave the campus, 
    once they realised I had been faking it all along. 
So, at the risk of bragging, I am proud of this thesis, 
    and the work it took to get here. 
\par
When you set out to write a thesis, 
    you are constantly reminded that no one will ever \emph{actually} read it, 
    and you must therefore decide how much time and effort to spend on a document to 
    gather dust on the electronic shelves of the university's digital library. 
I spent more time than most -- 
    not for the sake of examiners, supervisors or hypothetical future students, 
    but for myself, as evidence that I can produce good work when I really try. 
\par 

In the following 200 pages, I list in (sometimes boring) detail almost everything I've learned 
    in the last four and a half years. 
I've learned something else important: 
    the people around matter, and you need to tell them that once in a while.
% Acknolwedgement sections can read like the acceptance speech for an award you've given yourself,
%     with the writer congratulating themself under the guise of humility. 
Here I just want to mention some people that have helped me along the way, 
    or have been kind enough to share their time with me in some capacity. 
% Besides, I have managed to resist quoting Shakespeare, Jorce or Homer Simpson at the start of each chapter, 
%     so I hope I can be forgiven two pages of self-congratulations here instead.
If the world is lucky, I will never again write anything this long, 
    so I will have to write a lifetime worth of acknowledgements here. 
I will start in Bristol and work my way home. 

% Bristol
The first person to thank has to be Raf, since this entire thesis builds on his initial idea.
Beyond giving me the base material to work on for my PhD, 
    he has also given me a huge amount of help and support over the last few years, 
    along with friendship. 
His passion for machine learning is matched only by his passion for everything else; 
    he is endlessly interested in all facets of science. 
I hope to live up to his example. 
\par 

Andreas was equally crucial to any achievements I can claim. 
Through countless late night messages, he guided everything about this work that succeeded. 
Several times I thought I had outwitted him, and understood a concept more than him -- 
    until a few hours later when I realised I was wrong and he had explained it to me days before. 
Thanks for sticking with it. 
\par 

To Anthony and the rest of the Laing gang (pronounced \emph{Leng geng}), 
    thanks for the wisdom, stories and beers.
The wider \emph{QETLabs} group is a real privilege to have been part of -- 
    it's been inspiring and terrifying to be surrounded by such talented people.
Thanks to Sebastian for carefully reading my chapter on his beloved NV centres.
Thanks to everyone involved for putting up with me. 
\par

The \emph{Quantum Engineering Centre for Doctoral Training} gave me an opportunity I 
    never thought I'd have, as well as a community to enjoy it with. 
The idea that I can call myself a scientist means a lot to me, 
    and simply would not have happened if the QECDT hadn't taken a chance on me -- 
    I will always appreciate it. 
Of course, the CDT peaked with cohort 3:
    thanks to Dave for being willing to share a house in our first year, 
    to Will for teaching me how to play squash (and beating me every time), 
    and Max for a recent phone call where we agreed that thesis acknowledgements should
    read like the acceptance speech to an \emph{Oscar} you've awarded yourself, and to fully embrace it. 
I am grateful to Jorge (and Holly) for thinking of me to run \emph{Quantum in the Summer} together,
    and to Konstantina for working most closely with me along the way.  
\par 

I moved to Bristol for a PhD, 
    but I surely wouldn't have stayed without building a good life here. 
I've enjoyed all the time spent cycling; playing Gaelic football, soccer and squash;
    learning mind games like Avalon, and trips to cinemas and restaurants with colleagues and friends. 
I always say -- you can easily fill six nights per week with wholesome, healthy activities, 
    without drinking. 
For the seventh night, however, you should hope to make friends like I did:
    Alex, Dom, Jake, Joel, Frazer, Reece, and all those mentioned already.
Patrick had the unique punishment of thesis-reader, office-mate and pub-goer 
    (I am not sure what past-life indiscretion he is atoning for, but I'm grateful for it as well). 
My happiest memories of this stage of my life will be sunny beer gardens on Friday afternoons with all of the above. 

% Dublin
While the last year of my PhD (and thesis writing) coincided with Covid-19 lockdowns, 
    one silver lining has been catching up with my friends from Dublin more regularly. 
It is a pleasure to lose money to you all in poker, and I look forward to losing in person. 
Niall read my attempt to explain quantum mechanics, 
    and was gracious enough not to berate me for it. 
James and John Mark put up with more than their fair share over a number of weekends. 
Evan has read every CV, paper, and now thesis, that I've written in the last eight years. 
If Roy Keane had for football the passion Evan has for physics, 
    Ireland would have won the 2002 World Cup.
\par 

% Trim
Bridging the gap between Trinity College Dublin and my hometown of Trim is the real hero of this story:
    William Rowan Hamilton. 
Perhaps, after another 150 years, another physicist from Trim might profitably use knowledge of 
    Hamiltonians on a working quantum computer.
\par 

Weekends spent with old friends are high on my list of favourite things. 
I am always pleased when someone goes out of their way to visit -- 
    thanks to anyone who has gotten on a plane or train for my sake. 
Special mention to \emph{Big Red} -- a lifelong friend with a talent for 
    meandering, all-encompassing conversation, and a unique outlook:
    I've heard many people say \emph{Would this make me rich/famous/fulfilled}, 
    you're the only one to ever ask \emph{Would this make me happy?}
\par 

Along with all of the great people above, 
    I am lucky to have a supportive family. 
I hope this thesis goes some way to answering your question, 
    \emph{what do you actually do?} 
To the next generation: 
    I hope one day in ten or fifteen years, you will find this thesis some rainy afternoon. 
If you ask about it, don't believe them if they tell you I was smart. 
Believe them if they that say I worked hard, but that I thoroughly enjoyed it -- it would not have been worth doing otherwise. 
\par 
More important than any thesis, these years will be remembered for 
    the short time we got to spend with Lola and Luca. 
I still don't have words elegant enough to lessen their loss -- I just wanted to acknowledge them here. 
\par 

To Mam and Dad, sincere thanks for everything. 
We know you're proud of us; 
    we're proud of you too. 
\par 
Finally, Emma has been constant across all aspects of my life.
I don't deserve the patience you've shown, especially this last year when I've been even more stressful, 
    but I will always appreciate it, and do my best to make it worth your while.  