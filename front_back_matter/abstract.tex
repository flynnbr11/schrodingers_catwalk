\chapter*{Abstract}
\addcontentsline{toc}{chapter}{Abstract}

% People have always been, and will always be, curious about the world around them;
%     recently we've called this practice \emph{science}. 
% In the name of science, we've discovered that the earth revolves around the sun, 
%     produced countless medicines, and built atomic bombs.
% At every stage in the history of science, we na\"ively believed we had reached the bottom;
%     the current incarnation of this belief is \emph{quantum mechanics},
%     which promises to describe nature's most delicate procedures by making assumptions nobody understands. 
% Another fundamental human endeavour has been the design and construction of machinery to expedite crucial tasks;
%     from prehistoric hunting tools, to farm equipment in the industrial revolution, 
%     and the ubiquity of information technologies through  the $20^{\textrm{th}}$ century. 

% \par 
% The $21^{\textrm{st}}$ century marks the culmination of these core efforts in the realisation of \emph{quantum technologies}:
%     devices which exploit quantum mechanical processes to achieve outcomes beyond the reach of classical machinery.

Quantum technologies exploit quantum mechanical processes to achieve outcomes beyond the reach of classical machinery.
One of their most promising applications is quantum simulation, 
    whereby particles, atoms and molecules can be examined thoroughly for the first time, 
    having been beyond the scope of even the most powerful supercomputers. 
\par 
\emph{Models} have been useful tools in understanding physical systems:
    these are mathematical structures encoding physical interactions,
    which allow us to predict how the system will behave under various conditions. 
Models of quantum systems are particularly difficult to design and test, 
    owing to the huge computational resources required to represent them accurately.
In this thesis, we introduce and develop an algorithm to characterise quantum systems efficiently, 
    by inferring a model consistent with their observed dynamics.
The \emph{Quantum Model Learning Agent} (QMLA) is an extensible framework which permits 
    the study of any quantum system of interest, 
    by combining quantum simulation with state of the art machine learning.
QMLA iteratively proposes candidate models and trains them against the target system,
    finally declaring a single model as the best representation for the system of interest.  
\par 

We describe QMLA and its implementation through open source software,
    before testing it under a series of physical scenarios.
First, we consider idealised theoretical systems in simulation, 
    verifying the core principles of QMLA. 
Next, we incorporate strategies for generating candidate models
    by exploiting the information QMLA has gathered to date;
    by incorporating a genetic algorithm within QMLA, 
    we explore vast spaces of valid candidate models, with QMLA reliably identifying the precise target model.
Finally, we apply QMLA to \emph{realistic} quantum systems, 
    including operating on experimental data measured from an electron spin in a nitrogen vacancy centre. 

\par 

QMLA is shown to be effective in all cases studied in this thesis;
    however, of greater interest is the platform it provides for examining quantum systems.
QMLA can aid engineers in configuring experimental setups, 
    facilitate calibration of near term quantum devices,
    and ultimately enable complete characterisation of natural quantum structures.
This thesis marks the beginning of a new line of research, 
    into automating the understanding of quantum mechanical systems.
% \emph{"All models are wrong but some are useful"} - George Box