\addcontentsline{toc}{chapter}{Overview and contributions}

This part details the algorithms which form the basis for the research conducted in this thesis. 
The corresponding software is a primary outcome of this thesis \cite{flynn2021Quantum, flynn2021QMLA, qmla_docs}.

\par 
\vspace{1cm}

\Cref{chapter:qhl} introduces \gls{qhl}, an algorithm for the optimisation of \gls{hamiltonian} parameters
    when the form of the model describing a system of interest is known. 
    This is not presented as new work, but rather as a bedrock for later discussions. 
    The analysis and figures presented in this chapter are unique to this thesis but do not necessarily offer novel insights. 
\par 
\vspace{1cm}
\Cref{chapter:qmla} builds upon \gls{qhl} by posing the question: 
    without assuming access to the model describing the target system, can we combine model training algorithms, 
    in particular \gls{qhl}, with model recovery methodologies, to learn the \gls{hamiltonian} model 
    governing the system, and hence uncover the physics of quantum systems. 
    This motivation leads to the \acrfull{qmla}: 
    a machine learning framework for reverse engineering models of quantum systems from their data.
    This protocol was initially devised by Dr. Raffaele Santagati, 
    and developed together with myself and Drs. Andreas Gentile, Stefano Paesani, Nathan Wiebe and Chris Granade. 
    The protocol has been published in \cite{gentile2020learning}, 
    and applied to numerous case studies, which are described in later Parts. 
\par
\vspace{1cm}

\Cref{chapter:sw} describes the implementation of \gls{qmla} through an open source software package. 
I was the principal designer and programmer of the codebase described, which constitutes a large portion of the output of my research. 
The results presented in \cref{part:theoretical_study}, \cref{part:experimental_study} are all achieved through this framework. 
