This Part details the algorithm which form the basis for the research conducted in this thesis. 
The software described is available at \cite{flynn2021QMLA, qmla_docs}.

\par 
\vspace{1cm}

\Cref{chapter:qhl} introduces \glsentryfull{qhl}, an algorithm for the optimisation of Hamiltonian parameters
    when the form of the model describing a system of interest is known. 
    This is not presented as new work, but rather as a bedrock for later sections. 
    The analysis/figures presented in this chapter are unique to this thesis but do not necessarily offer novel insights. 
\par 
\vspace{1cm}
\Cref{chapter:qmla} builds upon \gls{qhl} by posing the question: 
    without assuming access to the model describing the target system, can we combine model training algorithms, 
    in particular \gls{qhl}, with model recovery methodologies, to learn the Hamiltonian model 
    governing the system, and hence uncover the physics of quantum systems. 
    This motivation leads to the \glsentryfull{qmla}: 
    a machine learning framework for reverse engineering models of quantum systems from the data it produces. 
    This protocol was initially devised by Dr. Rafaelle Santagati, with Drs. Andreas Gentile, Nathan Wiebe and Chris Granade. 
    I contributed to the refinement of \gls{qmla} with Drs. Santagati and Gentile; 
    the version presented in \cref{chapter:qmla} represents the culmination of several conceptual stages. 
    The protocol has been described in \cite{gentile2020learning}, which are described in later Parts. 
\par
\vspace{1cm}

\Cref{chapter:sw} describes the implementation of \gls{qmla} through an open source software package. 
I was the principle designer and programmer of the codebase described, which constitutes a large proportion of the output of my research. 
The results presented in \cref{part:theoretical_study}, \cref{part:experimental_study} are all found using this framework. 
