
We briefly summarise some alternative \gls{edh}s which we will use later in this thesis. 
\subsubsection{Forced exponential increase}
One early candidate for a viable \gls{edh} was 
\begin{equation}
    t_k = \bk{\frac{9}{8}}^k, 
\end{equation}
    where $k$ is the experiment number. 
This forces \gls{qhl} to consider exponentially-increasing times, irrespective of its current state of knowledge. 

\subsubsection{Fixed set}
By providing a list of times, chosen in advance, we can attempt to ensure that \gls{qhl} tries to account for 
    an entire dataset. 
For instance, if only a small set of experimental measurements are available, 
    it is sensible to train on all of them, perhaps repeatedly. 

\subsubsection{Random time upper bounded}
    Given $t_{max}$, time is chosen randomly: $t_k = \mathcal{U}(0, t_{max})$. 
    
\subsubsection{Mixed: PGH and fixed set}
In cases where we wish to allow \gls{smc} to train both adaptively and according to a fixed schedule,
    we can employ this \gls{edh}. 
For instance, where a small set of experimental data is available offline, 
    we can first use the adapative routine (\gls{pgh}) to coarsely refine the distribution, 
    but if the parameters cannot be learned sufficiently, e.g. because of noise in the data, 
    the posterior will never be narrow enough to allow the \gls{pgh} to reach higher 
    experimentally measured times. 
In this case, it is reasonable to force the \gls{edh} to focus some of the learning 
    resources on the data for higher times, 
    especially where much of the underlying physics of interest only starts to be visible for those times, 
    e.g. where the experimental system starts to decohere. 

\subsubsection{Altered \gls{pgh}}
A straightforward alternative is simply to replace the reciprocal in \cref{eqn:pgh_time} with any another distance metric, 
    $d(\al_i, \al_j)$.
In doing so we retain much of the reasoning for \gls{pgh}. 
While \gls{pgh} uses the Euclidean distance, we can instead consider, for example, 
    any of the distance metrics available in \ttt{scipy.spatial.distance.pdist} \cite{2020SciPy-NMeth}, 
    of which in particular we implement \ttt{cityblock, euclidean, chebyshev, canberra, braycurtis, minkowski}. 
        
\subsubsection{Volume adaptive \gls{pgh}}
Where it is suspected that \gls{pgh} is performing poorly, it is feasible to force the \gls{edh} to consider orders of magnitude 
    greater or smaller than the distance $d(\al_i, \al_j)$.
We include a heuristic which tracks the change in volume across experiments, 
    given by $\Delta V = 1 - \nicefrac{V^{\prime}}{V}$, where $V$ is the volume of the prior
    and $V^{\prime}$ is the volume of the posterior, 
    noting that several experiments may elapse between measuring $V, V^{\prime}$,
    such that we can interpret
\begin{align}
    \label{eqn:volume_adaptive_cases}
    \begin{cases}
        V = V^{\prime} \bk{ \textrm{no change} }  & \Delta V = 0   \\
        V^{\prime} > V  \bk{ \textrm{disimprovement} }  & \Delta V < 0  \\ 
        V^{\prime} < V \bk{ \textrm{improvement} }   & 0 < \Delta V < 1\\ 
    \end{cases}
\end{align}

\cref{eqn:volume_adaptive_cases} allows us to set acceptable levels for the learning via the volume reduction: 
    when the change in volume is deemed inappropriate, we force the experimental times into another order of magnitude, 
    by imposing a scalar factor on the \gls{pgh} chosen time, again facilitating any distance metric, $d$, 
\begin{equation}
    \label{eqn:volume_adaptive_pgh}
    t_k = \kappa_k \frac{1}{d(\al_i, \al_j)},
\end{equation}
    where $\kappa_k$ is a scalar set for each experiment $k$, initially set as $\kappa_0=1$.
For example, if $\Delta V > 0.2$ across a small number of experiments (say 10), 
    we see that the volume has reduced by $20\%$ over 10 experiments, which is a strong rate of learning, 
    so we retain $\kappa_k = \kappa_{k-1}$. 
Conversely, $\Delta V = 0.01$ indicates that the learning has stalled, 
    to compensate, we set $\kappa_{k} = 10 \kappa_{k-1}$, 
    such that \gls{smc} challenges $\pra$ to explain much more challenging experiments.
Likewise, in the case where $\pra$ disimproves significantly, say $\Delta V < -0.2$, 
    we can force the heuristic to design conservative experiments by setting $\kappa_k = \nicefrac{\kappa_{k-1}}{10}$. 
% Q: is this silly? does loschmidt echo mean high times can never be learned from?

\subsubsection{Sampled order of magnitude}
When a parameter distribution contains parameters of varying magnitude, 
    it is challenging to select a time that suitably trains all parameters simulataneously. 
One attempt to resolve this problem is to treat the parameters independently, 
    effectively trying to train them separately. 
We sample from the individual uncertainty on each parameter, 
    approximated as the square root of the covariance matrix's diagnoal element 
    corresponding to that parameter. 
For example, suppose a three parameter distribution involves $kHz, MHz$ and $GHz$ terms:
    the diagonal of the covariance matrix might read $\irow{10^{8}, 10^{27}, 10^{85}}$, 
    such that the individual uncertainties are $\irow{10^{2.8}, 10^{9}, 10^{9.2}}$.
Recognising that parameters of higher order of magnitude have a greater impact on dynamics, 
    we assign sampling probability of focusing on that parameter scale according to $log_{10}$ of the individual uncertainties. 
Then, the probabilities are $\{\frac{2.8}{21}, \frac{9}{21}, \frac{9.2}{21} \}$ for the $kHz, MHz, GHz$ terms respectively. 
Then, when a parameter scale is chosen, we take the distance between two sampled particles $\al_i, \al_j$ (as in \gls{pgh}),
    only with respect to that scale, e.g. if $MHz$ is sampled, $t_k \sim \mathcal{O}(10^{-6}s)$.
