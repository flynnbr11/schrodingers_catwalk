Most of the figures presented in the main text are generated directly by the \gls{qmla} framework. 
Here we list the implementation details of each figure so they may be reproduced
    by ensuring the configuration in \cref{table:figure_reproduction} are set in the launch script. 
The default behaviour of \gls{qmla} is to generate a results folder uniquely identified by the date and time 
    the \gls{run} was launched, e.g. results can be found at the \emph{\gls{results directory}} \ttt{qmla/Launch/Jan\_01/12\_34}. 
Given the large number of plots available, ranging from high-level \gls{run} perspective 
    down to the training of individual models, 
    we introduce a \ttt{plot\_level} $\in \{1, ..., 6\}$  for each \gls{run} of \gls{qmla}:
    higher \ttt{plot\_level} informs \gls{qmla} to generate more plots.  
\par  

Within the \gls{results directory}, the outcome of the run's \glspl{instance} are stored, 
    with analysis plots broadly grouped as
\begin{easylist}
    & \ttt{evaluation}: plots of probes and times used as the evaluation dataset. 
    & \ttt{single\_instance\_plots}: outcomes of an individual \gls{qmla} \gls{instance}, 
        grouped by the \gls{instance} ID. 
        Includes results of training of individual models (in \ttt{model\_training}), 
        as well as sub-directories for anlaysis at the branch level (in \ttt{branches}) and \ttt{comparisons}. 
    & \ttt{combined\_datasets}: 
        \ttt{pandas} dataframes containing most of the data used during analysis of the \gls{run}. 
        Note that data on the individual model/instance level may be discarded so some minor analyses can not be 
        performed offline. 
    & \ttt{exploration\_strategy\_plots} plots specifically required by the \gls{es} at the \gls{run} level.
    & \ttt{champion\_models}: analysis of the models deemed champions by at least one \gls{instance} in the \gls{run}, 
        e.g. average parameter estimation for a model which wins multiple instances. 
    & \ttt{performance}: evaluation of the \gls{qmla} \gls{run}, 
        e.g. the  \gls{win rate}  of each model and the number of times each term is found in \glspl{champion model}. 
    & \ttt{meta} analysis of the algorithm' implementation, e.g. timing of jobs on each process in a cluster; 
        generally users need not be concerned with these. 
\end{easylist}    
\par


In order to produce the results presented in this thesis, 
    the configurations listed in \cref{table:figure_reproduction} were input to the launch script.
The launch scripts in the \gls{qmla} codebase consist of many configuration settings for running \gls{qmla};
    only the lines in snippet in \cref{listing:launch_script} need to be set according to 
    altered to retrieve  the corresponding figures.
Note that the runtime of \gls{qmla} grows quite quickly with $\Ne, \Np$ (except for the \ttt{AnalyticalLikelihood} \gls{es}), 
    especially for the entire \gls{qmla} algorithm; running \gls{qhl} is feasible on a personal computer in 
    $<30$ minutes for $\Ne=1000; \Np=3000$. 

\begin{lstlisting}[
    label=listing:launch_script,
    caption=QMLA Launch scipt
]
#!/bin/bash

###############
# QMLA run configuration
###############
num_instances=1
run_qhl=1 # perform QHL on known (true) model
run_qhl_mulit_model=0 # perform QHL for defined list of models.
exp=200 # number of experiments
prt=1000 # number of particles

###############
# QMLA settings
###############
plot_level=6
debug_mode=0

###############
# Choose an exploration strategy 
###############

exploration_strategy='AnalyticalLikelihood'

\end{lstlisting}


\renewcommand{\arraystretch}{1.25} % space between rows
\setlength{\tabcolsep}{5pt}

\clearpage
\begin{table}[h!]
% \begin{sidewaystable}
    \begin{center}
        \begin{tabular}{lllll}
\hline
                                 &                                   &                                    $N_E$ &                                    $N_P$ &                                     Data \\
Figure & Exploration Strategy &                                          &                                          &                                          \\
\midrule
\multirow{4}{*}{\cref{fig:heuristics_test}} & \texttt{DemoHeuristicPGH} &                                     1000 &                                     3000 &                           Nov\_27/19\_39 \\
                                 & \texttt{DemoHeuristicNineEighths} &                                     1000 &                                     3000 &                           Nov\_27/19\_40 \\
                                 & \texttt{DemoHeuristicTimeList} &                                     1000 &                                     3000 &                           Nov\_27/19\_42 \\
                                 & \texttt{DemoHeuristicRandom} &                                     1000 &                                     3000 &                           Nov\_27/19\_47 \\
\cline{1-5}
\multirow{4}{*}{\cref{fig:probes_test}} & \texttt{DemoProbesPlus} &                                     1000 &                                     3000 &                           Nov\_27/14\_43 \\
                                 & \texttt{DemoProbesZero} &                                     1000 &                                     3000 &                           Nov\_27/14\_45 \\
                                 & \texttt{DemoProbesTomographic} &                                     1000 &                                     3000 &                           Nov\_27/14\_46 \\
                                 & \texttt{DemoProbes} &                                     1000 &                                     3000 &                           Nov\_27/14\_47 \\
\cline{1-5}
\cref{fig:param_learning_vary_particles} & \texttt{AnalyticalLikelihood} &                                      500 &                                     2000 &                           Nov\_16/14\_28 \\
\cref{fig:ising_two_param_learning} & \texttt{DemoIsing} &                                      500 &                                     5000 &                           Nov\_18/13\_56 \\
\cref{fig:ising_fully_parameterised} & \texttt{DemoIsing} &                                     1000 &                                     5000 &                           Nov\_18/13\_56 \\
\cref{fig:ising_model_types_dynamics} & \texttt{DemoIsing} &                                     1000 &                                     5000 &                           Nov\_18/13\_56 \\
\cline{1-5}
\multirow{3}{*}{\cref{fig:lattice_qmla_eg}} & \texttt{IsingLatticeSet} &                                     1000 &                                     4000 &                           Nov\_19/12\_04 \\
                                 & \texttt{IsingLatticeSet} &                                     1000 &                                     4000 &                           Nov\_19/12\_04 \\
                                 & \texttt{IsingLatticeSet} &                                     1000 &                                     4000 &                           Nov\_19/12\_04 \\
\cline{1-5}
\multirow{3}{*}{\cref{fig:lattice_success_rates}} & \texttt{IsingLatticeSet} &                                     1000 &                                     4000 &                           Sep\_30/22\_40 \\
                                 & \texttt{HeisenbergLatticeSet} &                                     1000 &                                     4000 &                           Oct\_22/20\_45 \\
                                 & \texttt{FermiHubbardLatticeSet} &                                     1000 &                                     4000 &                           Oct\_02/00\_09 \\
\cline{1-5}
\multirow{12}{*}{\cref{fig:family_classification}} & \texttt{IsingReducedLatticeSet} &                                      125 &                                      500 &                           Feb\_16/09\_12 \\
                                 & \texttt{HeisenbergReducedLatticeSet} &                                      125 &                                      500 &                           Feb\_16/09\_14 \\
                                 & \texttt{HubbardReducedLatticeSet} &                                      125 &                                      500 &                           Feb\_16/09\_16 \\
                                 & \texttt{IsingReducedLatticeSet} &                                      250 &                                     1000 &                           Feb\_15/21\_49 \\
                                 & \texttt{HeisenbergReducedLatticeSet} &                                      250 &                                     1000 &                           Feb\_15/21\_47 \\
                                 & \texttt{HubbardReducedLatticeSet} &                                      250 &                                     1000 &                           Feb\_15/21\_45 \\
                                 & \texttt{IsingReducedLatticeSet} &                                      500 &                                     2000 &                           Feb\_16/09\_20 \\
                                 & \texttt{HeisenbergReducedLatticeSet} &                                      500 &                                     2000 &                           Feb\_16/09\_19 \\
                                 & \texttt{HubbardReducedLatticeSet} &                                      500 &                                     2000 &                           Feb\_16/09\_18 \\
                                 & \texttt{IsingReducedLatticeSet} &                                     1000 &                                     4000 &                           Feb\_16/18\_33 \\
                                 & \texttt{HeisenbergReducedLatticeSet} &                                     1000 &                                     4000 &                           Feb\_16/18\_34 \\
                                 & \texttt{HubbardReducedLatticeSet} &                                     1000 &                                     4000 &                           Feb\_16/18\_35 \\
\hline
\end{tabular}

    \end{center}
    \caption[Figure implementation details]{Implementation details for figures used in the main text.}
    \label{table:figure_reproduction}
% \end{sidewaystable}
\end{table}

\clearpage
\begin{table}[h!]
        \begin{center}
            \begin{tabular}{lllll}
\hline
                        &                                                      &                                    $N_E$ &                                    $N_P$ &                                     Data \\
Figure & Exploration Strategy &                                          &                                          &                                          \\
\midrule
\cref{fig:ga_param_sweep} & \texttt{N/A} &                                      N/A &                                      N/A &                           Dec\_07/22\_04 \\
\cline{1-5}
\multirow{4}{*}{\cref{fig:bf_by_fscore}} & \texttt{DemoBayesFactorsByFscore} &                                      500 &                                     2500 &                           Dec\_09/12\_29 \\
                        & \texttt{DemoFractionalResourcesBayesFactorsByFscore} &                                      500 &                                     2500 &                           Dec\_09/12\_31 \\
                        & \texttt{DemoBayesFactorsByFscore} &                                     1000 &                                     5000 &                           Dec\_09/12\_33 \\
                        & \texttt{DemoBayesFactorsByFscoreEloGraphs} &                                      500 &                                     2500 &                           Dec\_09/12\_32 \\
\cline{1-5}
\cref{fig:single_models_elo_ratings} & \texttt{HeisenbergGeneticXYZ} &                                      500 &                                     2500 &                           Dec\_10/14\_40 \\
\cline{1-5}
\multirow{2}{*}{\cref{fig:single_generation_all_ratings}} & \texttt{HeisenbergGeneticXYZ} &                                      500 &                                     2500 &                           Dec\_10/14\_40 \\
                        & \texttt{HeisenbergGeneticXYZ} &                                      500 &                                     2500 &                           Dec\_10/14\_40 \\
\cline{1-5}
\multirow{2}{*}{\cref{fig:ga_instance}} & \texttt{HeisenbergGeneticXYZ} &                                      500 &                                     2500 &                           Mar\_07/12\_40 \\
                        & \texttt{HeisenbergGeneticXYZ} &                                      500 &                                     2500 &                           Dec\_10/16\_12 \\
\cline{1-5}
\cref{fig:ga_run} & \texttt{HeisenbergGeneticXYZ} &                                      500 &                                     2500 &                           Dec\_18/20\_12 \\
\cline{1-5}
\multirow{2}{*}{\cref{fig:nv_model_composition}} & \texttt{NVCentreExperimentalData} &                                     1000 &                                     3000 &                      2019/Oct\_02/18\_01 \\
                        & \texttt{SimulatedExperimentNVCentre} &                                     1000 &                                     3000 &                      2019/Oct\_02/18\_16 \\
\cline{1-5}
\cref{fig:nv_model_dynamics} & \texttt{NVCentreExperimentalData} &                                     1000 &                                     3000 &                      2019/Oct\_02/18\_01 \\
\cref{fig:nv_model_composition} & \texttt{SimulatedExperimentNVCentre} &                                     1000 &                                     3000 &                      2019/Oct\_02/18\_16 \\
\cline{1-5}
\multirow{2}{*}{\cref{fig:nv_learned_params}} & \texttt{SimulatedExperimentNVCentre} &                                     1000 &                                     3000 &                      2019/Oct\_02/18\_16 \\
                        & \texttt{NVCentreExperimentalData} &                                     1000 &                                     3000 &                      2019/Oct\_02/18\_01 \\
\cline{1-5}
\multirow{2}{*}{\cref{fig:nv_ga_eval_data}} & \texttt{NVCentreGenticAlgorithmPrelearnedParameters} &                                        2 &                                        5 &                           Sep\_09/12\_00 \\
                        & \texttt{NVCentreGenticAlgorithmPrelearnedParameters} &                                        2 &                                        5 &                           Sep\_09/12\_00 \\
\cline{1-5}
\multirow{2}{*}{\cref{fig:nv_ga_instance}} & \texttt{NVCentreGenticAlgorithmPrelearnedParameters} &                                        2 &                                        5 &                           Sep\_09/12\_00 \\
                        & \texttt{NVCentreGenticAlgorithmPrelearnedParameters} &                                        2 &                                        5 &                           Sep\_09/12\_00 \\
\cline{1-5}
\cref{fig:nv_ga_run_models} & \texttt{NVCentreGenticAlgorithmPrelearnedParameters} &                                        2 &                                        5 &                           Sep\_08/23\_58 \\
\cref{fig:nv_ga_hinton} & \texttt{NVCentreGenticAlgorithmPrelearnedParameters} &                                        2 &                                        5 &                           Sep\_08/23\_58 \\
\hline
\end{tabular}

        \end{center}
        \caption[Figure implementation details continued]{[Continued from \cref{table:figure_reproduction}] Implementation details for figures used in the main text.}
        \label{table:figure_reproduction_contd}
\end{table}
    
\clearpage