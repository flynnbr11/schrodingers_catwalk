Most of the figures presented in the main text are generated directly by the \gls{qmla} framework. 
Here we list the implementation details of each figure so they may be reproduced
    by ensuring the configuration in \cref{table:figure_reproduction} are set in the launch script. 
The default behaviour of \gls{qmla} is to generate a results folder uniquely identified by the date and time 
    the \gls{run} was launched, e.g. results can be found at the \emph{results directory} \ttt{qmla/Launch/Jan\_01/12\_34}. 
Given the large number of plots available, ranging from high-level run perspective 
    down to the training of individual models, 
    we introduce a \ttt{plot\_level} $\in \{1, ..., 6\}$  for each \gls{run} of \gls{qmla}:
    higher \ttt{plot\_level} informs \gls{qmla} to generate more plots.  
\par  

Within the results directory, the outcome of the run's instances are stored, 
    with analysis plots broadly grouped as
\begin{easylist}
    & \ttt{evaluation}: plots of probes and times used as the evaluation dataset. 
    & \ttt{single\_instance\_plots}: outcomes of an individual \gls{qmla} \gls{instance}, 
        grouped by the instance ID. 
        Includes results of training of individual models (in \ttt{model\_training}), 
        as well as sub-directories for anlaysis at the branch level (in \ttt{branches}) and \ttt{comparisons}. 
    & \ttt{combined\_datasets}: 
        \ttt{pandas} dataframes containing most of the data used during analysis of the \gls{run}. 
        Note that data on the individual model/instance level may be discarded so some minor analyses can not be 
        performed offline. 
    & \ttt{exploration\_strategy\_plots} plots specifically required by the \gls{es} at the \gls{run} level.
    & \ttt{champion\_models}: analysis of the models deemed champions by at least one \gls{instance} in the \gls{run}, 
        e.g. average parameter estimation for a model which wins multiple instances. 
    & \ttt{performance}: evaluation of the \gls{qmla} \gls{run}, 
        e.g. the win rate of each model and the number of times each term is found in champion models. 
    & \ttt{meta} analysis of the algorithm' implementation, e.g. timing of jobs on each process in a cluster; 
        generally users need not be concerned with these. 
\end{easylist}    
\par


In order to produce the results presented in this thesis, 
    the configurations listed in \cref{table:figure_reproduction} were input to the launch script.
The launch scripts in the \gls{qmla} codebase consist of many configuration settings for running \gls{qmla};
    only the lines in snippet in \cref{listing:launch_script} need to be set according to 
    altered to retrieve  the corresponding figures.
Note that the runtime of \gls{qmla} grows quite quickly with $N_E, N_P$ (except for the \ttt{AnalyticalLikelihood} \gls{es}), 
    especially for the entire \gls{qmla} algorithm; running \gls{qhl} is feasible on a personal computer in 
    $<30$ minutes for $\Ne=1000; \Np=3000$. 

\begin{lstlisting}[
    label=listing:launch_script,
    caption="QMLA Launch scipt"
]
#!/bin/bash

###############
# QMLA run configuration
###############
num_instances=1
run_qhl=1 # perform QHL on known (true) model
run_qhl_mulit_model=0 # perform QHL for defined list of models.
exp=200 # number of experiments
prt=1000 # number of particles

###############
# QMLA settings
###############
plot_level=6
debug_mode=0

###############
# Choose an exploration strategy 
###############

exploration_strategy='AnalyticalLikelihood'

\end{lstlisting}

\clearpage
% \begin{table}[h!]
\begin{sidewaystable}
    \begin{center}
        \begin{tabular}{lllrrl}
\hline
                        &                                            &                                Algorithm &                                    $N_E$ &                                    $N_P$ &                                     Data \\
Figure & Exploration Strategy &                                          &                                          &                                          &                                          \\
\midrule
\cref{fig:param_learning_vary_particles} & \texttt{AnalyticalLikelihood} &                                      QHL &                                      500 &                                     2000 &                           Nov\_16/14\_28 \\
\cref{fig:ising_two_param_learning} & \texttt{DemoIsing} &                                      QHL &                                      500 &                                     5000 &                           Nov\_18/13\_56 \\
\cref{fig:ising_fully_parameterised} & \texttt{DemoIsing} &                                      QHL &                                     1000 &                                     5000 &                           Nov\_18/13\_56 \\
\cref{fig:ising_model_types_dynamics} & \texttt{DemoIsing} &                                      QHL &                                     1000 &                                     5000 &                           Nov\_18/13\_56 \\
\cref{fig:lattice_qmla_eg} & \texttt{IsingLatticeSet} &                                     QMLA &                                     1000 &                                     4000 &                           Nov\_19/12\_04 \\
\multirow{3}{*}{\cref{fig:lattice_success_rates}} & \texttt{IsingLatticeSet} &                                     QMLA &                                     1000 &                                     4000 &                           Sep\_30/22\_40 \\
                        & \texttt{HeisenbergLatticeSet} &                                     QMLA &                                     1000 &                                     4000 &                           Oct\_22/20\_45 \\
                        & \texttt{FermiHubbardLatticeSet} &                                     QMLA &                                     1000 &                                     4000 &                           Oct\_02/00\_09 \\
\cline{1-6}
\multirow{4}{*}{\cref{fig:bf_by_fscore}} & \texttt{DemoBayesFactorsByFscore} &                                     QMLA &                                      500 &                                     2500 &                           Dec\_09/12\_29 \\
                        & \texttt{DemoFractionalResourcesBayesFactorsByFscore} &                                     QMLA &                                      500 &                                     2500 &                           Dec\_09/12\_29 \\
                        & \texttt{DemoBayesFactorsByFscore} &                                     QMLA &                                     1000 &                                     5000 &                           Dec\_09/12\_29 \\
                        & \texttt{DemoBayesFactorsByFscoreEloGraphs} &                                     QMLA &                                      500 &                                     2500 &                           Dec\_09/12\_29 \\
\hline
\end{tabular}

    \end{center}
    \caption[Figure implementation details]{Implementation details for figures used in the main text.}
    \label{table:figure_reproduction}
\end{sidewaystable}
% \end{table}
\clearpage