This Part details the application of the algorithms described in \cref{part:algorithms}
    to the study of experimental and realistic quantum systems.
Arguments here were presented in \cite{gentile2020learning}. 
\par 
\vspace{1cm}

In \cref{chapter:nv}, we describe the physical system considered, 
    the \glsentryfull{nv} centre in diamond. 
We detail the design of an \glsentrylong{es} within the \glsentrylong{qmla} framework
    targetting the study of such a system. 
The application was conceived by Dr. Rafaelle Santagati; 
    the retrieval of experimental data used throughout this section, 
    as well as the initial model reduction to a set of sensible Hamilontonian terms,
    were performed by Drs. Sebastian Knauer and Andreas Gentile. 
The machine learning methodologies presented, 
    such as the greedy search rule, were refined by Drs. Santagati, Gentile and myself. 
I performed the adaptation of the \gls{qmla} software, 
    ran the instances, analysed the data and generated the figures, except where explicitly referenced.
\par 
\vspace{1cm}

\Cref{chapter:many_qubits} continues the theme of applying \gls{qmla} to data from realistic systems: 
    we extend the analysis to larger systems than those considered in \cref{chapter:nv}, 
    at the expense of resorting to simulations instead of experimental data. 
I proposed genetic algorithms for the exploration of large model spaces within \gls{qmla}, 
    as examined in \cref{part:theoretical_study}, including the study presented in this chapter. 
Together with Drs. Knauer, Gentile, Santagati and Nathan Wiebe, we devised the target model, 
    including the choice of parameters, to reflect a realistic system 
    interacting with a spin-bath environment.
I adapted the \gls{qmla} software, ran the instances, performed the analysis and generated the figures 
    shown in this chapter. 


