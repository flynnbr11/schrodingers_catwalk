\newcommand{\newdefinedabbreviation}[5]{
    \newglossaryentry{#1}{
        text={#2},
        long={#3},
        name={\glsentrylong{#1} (\glsentrytext{#1})},
        first={\glsentryname{#1}},
        firstplural={\glsentrylong{#1}\glspluralsuffix (\glsentryname{#1}\glspluralsuffix )},
        description={#4}
    }
}

%%%%%%%%%%% ACRONYMS %%%%%%%%%%%
\newacronym[
    description={Quantum Model Learning Agent, \cref{chapter:qmla}.}
]{qmla}{QMLA}{Quantum Model Learning Agent}
\newacronym[
    description={quantum mechanics, \cref{chapter:qm}}
]{qm}{QM}{quantum mechanics}
\newacronym[
    description={quantum Hamiltonian learning, \cref{chapter:qhl}.}
]{qhl}{QHL}{quantum Hamiltonian learning}
\newacronym[
    description={Bayes factor, \cref{sec:bayes_factors}}
]{bf}{BF}{Bayes factor}
\newacronym{nisq}{NISQ}{noisy intermediate scale quantum}
\newacronym[
    description={machine learning, \cref{chapter:ml}}
]{ml}{ML}{machine learning}
\newacronym{rl}{RL}{reinforcement learning}
\newacronym{ai}{AI}{artificial intelligence}
\newacronym[
    description={quantum machine learning, \cref{sec:quantum_machine_learning}}
]{qml}{QML}{quantum machine learning}
\newacronym{nv}{NV}{nitrogen-vacancy}
\newacronym[
    description={nitrogen-vacancy centre, \cref{sec:nv_centres}}
]{nvc}{NVC}{nitrogen-vacancy centre}
\newacronym[
    description={quadratic loss, \cref{eqn:quadratic_loss}}
]{ql}{QL}{quadratic loss}
\newacronym[
    description={genetic algorithm, \cref{sec:genetic_algorithms}}
]{ga}{GA}{genetic algorithm}
\newacronym{ges}{GES}{genetic exploration strategy}
\newacronym{gr}{GR}{growth rule}
\newacronym{fh}{FH}{Fermi-Hubbard}
\newacronym{fhm}{FHM}{Fermi-Hubbard model}
\newacronym{im}{IM}{Ising model}
\newacronym{hm}{HM}{Heisenberg model}
\newacronym[
    description={Bayes factor enhanced Elo ratings, \cref{sec:elo}}
]{bfeer}{BFEER}{Bayes factor enhanced Elo ratings}
\newacronym{gagr}{GAGR}{genetic algorithm growth rule}
\newacronym[
    description={sequential Monte Carlo, \cref{sec:smc}}
]{smc}{SMC}{sequential Monte Carlo}
\newacronym[
    description={total log-likelihood}
]{tll}{TLL}{total log-likelihood}
\newacronym[
    description={log total likelihood, \cref{sec:total_log_total_likelihood}}
]{ltl}{LTL}{log total likelihood}
\newacronym[
    description={total log total likelihood, \cref{eqn:log_total_likelihood}}
]{tltl}{TLTL}{total log total likelihood}
\newacronym{aic}{AIC}{Akaike information criterion}
\newacronym{aicc}{AICC}{Akaike information criterion corrected}
\newacronym{bic}{BIC}{Bayesian information criterion}
\newacronym[
    description={log-likelihood}
]{ll}{LL}{log-likelihood}
\newacronym[
    description={true positives, \cref{table:classification_metrics}}
]{tp}{TP}{true positives}
\newacronym[
    description={true negatives, \cref{table:classification_metrics}}
]{tn}{TN}{true negatives}
\newacronym[
    description={false positives, \cref{table:classification_metrics}}
]{fp}{FP}{false positives}
\newacronym[
    description={false negatives, \cref{table:classification_metrics}}
]{fn}{FN}{false negatives}
\newacronym{svm}{SVM}{support vector machine}
\newacronym[
    description={interactive quantum likelihood estimation, \cref{sec:iqle}}
]{iqle}{IQLE}{interactive quantum likelihood estimation}
\newacronym[
    description={quantum likelihood estimation, \cref{sec:likelihood}}
]{qle}{QLE}{quantum likelihood estimation}
\newacronym[
    description={classical likelihood estimation, \cref{sec:likelihood}}
]{cle}{CLE}{classical likelihood estimation}
\newacronym{hpd}{HPD}{high particle density}
\newacronym{mvee}{MVEE}{minimum volume enclosing ellipsoid}
\newacronym[
    description={experiment design heuristic, \cref{sec:heuristic}}
]{edh}{EDH}{experiment design heuristic}
\newacronym[
    description={particle guess heuristic, \cref{sec:pgh}}
]{pgh}{PGH}{particle guess heuristic}
\newacronym[
    description={exploration tree, \cref{sec:qmla_protocol}}
]{et}{ET}{exploration tree}
\newacronym[
    description={exploration strategy, \cref{sec:exploration_strategies}}
]{es}{ES}{exploration strategy}
\newacronym[
    description={objective function, \cref{sec:performance_metrics}}
]{of}{OF}{objective function}
\newacronym{carbon}{$C$}{carbon}
\newacronym{nitrogen}{$^{14}N$}{nitrogen-14}
\newacronym{pl}{PL}{photoluminescence}
\newacronym{qma}{QMA}{Quantum Merlin Arthur}
\newacronym{np}{NP}{Nondeterministic Polynomial}
\newacronym{gpu}{GPU}{graphics processing unit}
\newacronym{cpu}{CPU}{central processing unit}
\newacronym{qc}{QC}{quantum computer}
\newacronym{nn}{NN}{neural network}
\newacronym{vqe}{VQE}{variational quantum eigensolver}
\newacronym{pbs}{PBS}{portable batch system}
\newacronym{mw}{MW}{microwave}
\newacronym{dag}{DAG}{directed acyclic graph}
\newacronym{jwt}{JWT}{Jordan Wigner transformation}
\newacronym[description={Loschmidt echo, \cref{sec:iqle}}]{le}{LE}{Loschmidt echo}

% \newacronym{le}{LE}{\gls{Loschmidt echo}}

%%%%%%%%%%% DEFINED ACRONYM %%%%%%%%%%%
%%%%%%%%%%% will appear in the glossary instead of acronyms section %%%%%%%%%%%


\newdefinedabbreviation{api}{API}{
    % macro taken from https://tex.stackexchange.com/questions/8946/how-to-combine-acronym-and-glossary
    Application Programming Interface}{
    An Application Programming Interface (API) is a particular set of rules and specifications that a 
    software program can follow to access and make use of the services and resources provided by another 
    particular software program that implements that API. 
}

\newdefinedabbreviation{test_acr}{QMLA}{
    % macro taken from https://tex.stackexchange.com/questions/8946/how-to-combine-acronym-and-glossary
    Test this structure}{
    Here is a description. 
}


%%%%%%%%%%% TERMS %%%%%%%%%%%
\newglossaryentry{expectation value}
{
    name=expectation value,
    description={Average outcome expected by measuring an observable of a quantum system many times, \cref{sec:expectation_value}.}
}

\newglossaryentry{model search}
{
    name=model search,
    description={Exploration through the \gls{model space} performed by \acrshort{qmla}, where the progression of the search is determined by the \acrshort{es}.}
}

\newglossaryentry{spawn}
{
    name=spawn,
    description={Process by which new models are generated within an \acrshort{es}, ususally by combining previously considered models.}
}

\newglossaryentry{likelihood}
{
    name=likelihood,
    description={Value that represents how likely a hypothesis is; usually used in the context of likelihood estiamation, \cref{sec:likelihood}.}
}
\newglossaryentry{probe}
{
    name=probe,
    description={Input state, $\ket{\psi}$, into which the target system is initialised, before unitary evolution.},
    plural={probes},
}
\newglossaryentry{volume}
{
    name=volume,
    description={Volume of a parameter distribution's credible region, \cref{sec:volume}.}
}
\newglossaryentry{hyperparameter}
{
    name=hyperparameter,
    description={Variable within an algorithm that determines how the algorithm itself proceeds.}
}
\newglossaryentry{model}
{
    name=model,
    description={The mathematical description of some quantum system, \cref{sec:models}.}
}
\newglossaryentry{model space}
{
    name=model space,
    description={Abstract space containing all descriptions (within defined constraints such as dimension) of the system as \glspl{model}.}
}
\newglossaryentry{win rate}
{
    name=win rate,
    description={For a given candidate model, the fraction of \glspl{instance} within a \gls{run} which nominated it as champion.}
}
\newglossaryentry{success rate}
{
    name=success rate,
    description={Fraction of \glspl{instance} within a \gls{run} where \acrshort{qmla} nominates the \gls{true model} as \gls{champion}.}
}
\newglossaryentry{true model}
{
    name={true model} ,
    description={$\ho$, the model which correctly describes the target system, \gls{q}. $\ho$ is known for simulated \gls{q} 
    but not known precisely for experimental systems.}
}

\newglossaryentry{run}
{
    name=run,
    description={Collection of \acrshort{qmla} \glspl{instance}, usually targeting the same system with the same initial conditions.}
}
\newglossaryentry{instance}
{
    name=instance,
    description={A single implementation of the \acrshort{qmla} algorithm, resulting in a nominated \gls{champion model}.}
}

\newglossaryentry{champion}
{
    name=champion,
    description={See \gls{champion model}.}
}

\newglossaryentry{champion model}
{
    name=champion model,
    description={The model deemed by \acrshort{qmla} as the most suitable for describing the target system.}
}
\newglossaryentry{q}
{
    name=$Q$,
    description={Quantum system which is the target of \acrshort{qmla}, i.e. the system to be characterised.}
}

\newglossaryentry{term}
{
    name=term,
    description={Individual constituent of a model, 
    e.g. a single operator within a sum of operators, which in total describe a Hamiltonian, \cref{sec:models}.
    }
}

\newglossaryentry{results directory}
{
    name={results directory},
    description={
        Directory to which the data and analysis for a given \gls{run} of \acrshort{qmla} are stored.
    }
}

\newglossaryentry{chromosome}
{
    name=chromosome,
    description={
        A single candidate, in the space of valid solutions to the posed problem in a genetic algorithm, \cref{sec:genetic_algorithms}.
    }
}

\newglossaryentry{gene}
{
    name=gene,
    description={
        Individual element within a \gls{chromosome}.
    }
}


%%%%%%%%%%% MAKE - keep at bottom of file %%%%%%%%%%%
\makeglossaries
% \setacronymstyle{long-short-desc}