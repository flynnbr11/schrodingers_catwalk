%%%%%%%%%%%%%%%%%%%%%%%%%%%%%%%%%%%%%%%
%%%%%%%%%%%%% PACKAGES %%%%%%%%%%%%%
%%%%%%%%%%%%%%%%%%%%%%%%%%%%%%%%%%%%%%%

%%%%%%%%%%%%% Thesis formatting %%%%%%%%%%%%%

\usepackage[parts,pdfspacing]{classicthesis}
\usepackage[a4paper, total={7in, 8.5in}]{geometry}

%%%%%%%%%%%%% Standard packages %%%%%%%%%%%%%
\usepackage[utf8]{inputenc}
\usepackage[english]{babel}
\usepackage{titlesec}
\usepackage{xr}
\usepackage{float}
\usepackage{graphicx}
\usepackage{subfiles}
\usepackage{datetime}
\usepackage[
    font=small,
]{caption}
\usepackage{subcaption}
\usepackage{amsmath}
\usepackage{mathtools}
\numberwithin{equation}{chapter}
% Algorithm writing
\usepackage[ruled]{algorithm2e} 
\usepackage{newfloat,algcompatible} 
\widowpenalty=10000
\clubpenalty=10000


\usepackage{listings}
\usepackage{amssymb}
\usepackage[ampersand, itemize]{easylist}
\usepackage[hidelinks]{hyperref}
\usepackage{enumitem}
 % TODO colour glossary items but don't allow large red box
\hypersetup{
  pdfborder = {0 0 0},
  colorlinks,
  citecolor=red,
  filecolor=black,
  linkcolor=black,
  linktoc=all,
  urlcolor=black
}

\newlist{postulate}{enumerate}{1}     % this creates a dedicated counter named 'subtaski'
\setlist[postulate,1]{label=\arabic*} % set form of enumeration label here

\usepackage[capitalise]{cleveref}
\crefname{postulatei}{postulate}{postulates}


% \crefname{algocf}{alg.}{algs.}
% \Crefname{algocf}{Algorithm}{Algorithm}

%%%%%%%%%%%%% Nonstandard packages %%%%%%%%%%%%%
\usepackage[
  style=long,
  % style=altlongragged4col, 
  nolist, 
  nonumberlist,
  nopostdot,
  acronym
]{glossaries}
% \setacronymstyle{long-short-desc}
% \setacronymstyle{long-short-desc}
\renewcommand{\glsnamefont}[1]{\textbf{#1}}
% \setlength\LTleft{0pt}
% \setlength\LTright{0pt}
\setlength\glsdescwidth{0.8\hsize}

% \makeglossaries

\usepackage[colorinlistoftodos,prependcaption,textsize=tiny]{todonotes}
\usepackage{braket}
\usepackage{amsthm}
\usepackage{nicefrac}
% \usepackage{subfig}
% \usepackage{underscore}
\usepackage{longtable}
\usepackage{multirow}


%%%%%%%%%%%%%%%%%%%%%%%%%%%%%%%%%%%%%%%
%%%%%%%%%%%%% Settings %%%%%%%%%%%%%
%%%%%%%%%%%%%%%%%%%%%%%%%%%%%%%%%%%%%%%

% Format figures, tables etc
\makeatletter
\@addtoreset{table}{chapter}
\makeatother
\renewcommand\thetable{\thechapter.\arabic{table}}

\makeatletter
\@addtoreset{figure}{chapter}
\makeatother
\renewcommand\thefigure{\thechapter.\arabic{figure}}

\renewcommand{\thefootnote}{\fnsymbol{footnote}}
\newcommand{\tabitem}{~~\llap{\textbullet}~~}
\usepackage{rotating}

% Ensure chapter/part numbers are legible and formatted 
\titleformat{\chapter}[display]  
{\relax}{\vspace*{-3\baselineskip}\makebox[\linewidth][r]{\color{halfgray}\chapterNumber\thechapter}}{0pt}%
{\raggedright\spacedallcaps}[\normalsize\vspace*{.8\baselineskip}\titlerule]%

\renewcommand{\thepart}{\Roman{part}}
%\AtBeginDocument{\renewcommand{\thepart}{\Roman{part}}}
\titleformat{\part}[display]{\normalfont\large\centering}{\thispagestyle{empty}\partname~\thepart}{1em}{\color{red}\spacedallcaps}
% \DeclareCaptionType{code}[Code Listing][List of Code Listings] 

% Listing formatting
\lstset{
  % numbers=left, % comment this line to remove line numbers from code listings
  numberstyle=\small, 
  numbersep=8,
  frame=single, 
  language=Python, 
  xleftmargin=.05\textwidth,
  xrightmargin=.02\textwidth,
  framexleftmargin=20pt,
  framextopmargin=6pt,
  framexbottommargin=6pt, 
  breaklines=TRUE,
  showstringspaces=false,
  captionpos=b,
  deletekeywords={NUMBER}
}
\renewcommand{\thefootnote}{\arabic{footnote}}

% Miscellaneous settings
\setcounter{secnumdepth}{4}

% \theoremstyle{plain}
% \newtheorem{theorem}{Theorem}[section]
% \newtheorem{lemma}[thm]{Lemma}
% \newtheorem{proposition}[thm]{Proposition}
% \newtheorem*{corollary}{Corollary}

% \theoremstyle{definition}
% \newtheorem{definition}{Definition}[section]
% \newtheorem{conjecture}{Conjecture}[section]
\newtheorem{example}{Example}[section]

% \theoremstyle{remark}
% \newtheorem*{remark}{Remark}
% \newtheorem*{note}{Note}
\usepackage{xcolor}

\theoremstyle{definition}
\newtheorem{definition}{Definition}[section]

\theoremstyle{remark}
\newtheorem*{remark}{Remark}

% \theoremstyle{definition}
% \newtheorem{definition}[thm]{Definition} % definition numbers are dependent on theorem numbers
% \theoremstyle{example}
% \newtheorem{example}[Example]{Example} % same for example numbers
