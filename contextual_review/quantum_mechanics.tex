The study of nature at the quantum level has been at the forefront of physics since the early $20^{th}$ century. 
\dots

Here we will only introduce concepts utilised in this thesis.
We elucidate some fundamental topics of linear algebra and quantum theory in \cref{apdx:fundamentals} for completeness, 
    but consider them too cumbersome to include in the main text. 
For a more complete and general introduction to \gls{qm}, the reader is referred to \cite{griffiths2018introduction}.

\section{QM TODO}
\begin{easylist}[itemize]
    & states
    && qubits
    & Hilbert space
    && pure/separable and mixed
    & operators/gates
    & Pauli matrices
    & Bloch sphere
    & measurement
    && projectors
    & expectation value
    & superposition/entanglement
    & Hamiltonian
    & Schrodinger equation
    & Unitary evolution
    & first/second quantisation
    & quantum tech other than computation/simulation
\end{easylist}

\section{Quantum Mechanics and Information}

At any time, a quantum system, $Q$, can be described by its \emph{wavefunction}, $\Psi(t)$, 
    which contains all information about $Q$. 
In analogy with Newton's second law of motion, 
    which allows for the determination of a particle's position at any time, $\vec{r}(t)$, 
    given its initial position, $\vec{r}(t_0)$, and conditions such as mass and acceleration, 
    quantum \emph{equations of motion} can describe the evolution of $Q$ through its wavefunction \cite{dirac1981principles}. 
One proposal\footnotemark \ for the equation of motion 
    to describe the evolution of the wavefunction under known conditions, 
    i.e. determining $\Psi(t)$ from $\Psi(t_0) \ \forall t$, 
    is \emph{\schrodinger's equation}  \cite{griffiths2018introduction, mart2020introduce, nelson1966derivation}.
\footnotetext{
    The most noteworthy alternative formalism, due to Heisenberg \cite{heisenberg1985quantentheoretische}, 
    was shown equivalent to the \schrodinger picture described here.
}
\par 

Although the \schrodinger equation is a \emph{postulate} of \gls{qm} (see \cref{sec:postulates}), 
    let us introduce it in reverse order to elucidate its meaning, following \cite{susskind2014quantum}. 
We consider wavefunctions using \emph{Dirac notation} (\cref{sec:dirac_notation}).
Suppose we have two such wavefunctions, $\ket{\phi(t)}, \ket{\psi(t)}$ which are functions of time $t > t_0$.
We start with the assumption that \emph{similarity} is conserved between two wavefunctions,
    if they undergo the same transformation 
    (Susskind's \emph{minus first} law of classical mehcanics \cite{susskind2014quantum})
\begin{equation}
    \label{eqn:conservation_simularity}
    \braket{\phi(t) | \psi(t)} = \braket{\phi(t_0) | \psi(t_0)}
\end{equation}

Then, assuming some equations of motion capture the dynamics of $Q$, 
    there exists some evolution operator, $\hat{U}(t)$, which deterministically maps $\ket{\phi(t_o)}$ to $\ket{\psi(t)}$.
\begin{equation}
    \label{eqn:state_at_t}
    \ket{\psi(t)} = \hat{U}(t) \ket{\psi(t_0)},
\end{equation}
    where we have not yet imposed any restrictions on $\hat{U}$. 
Combining \crefrange{eqn:conservation_simularity}{eqn:state_at_t}, 
\begin{equation}
    \begin{split}
        \braket{\phi(t) | \psi(t)} &= \braket{ \phi(t_0) | \hat{U}^{\dag} \hat{U}| \psi(t_o) }
        \\
        \Rightarrow \braket{ \phi(t_0) | \hat{U}^{\dag}(t) \hat{U}(t) | \psi(t_o) } &= \braket{\phi(t_0) | \psi(t_0)}
        \\
        \Rightarrow \hat{U}^{\dag}(t) \hat{U}(t) &= \ident \ \ \ \ \forall t,
    \end{split}
\end{equation}
where the result $\hat{U}^{\dag} \hat{U} = \ident$ is the condition for \emph{unitarity} (\cref{sec:unitary}), 
    so we can claim the quantum wavefunction evolves unitarily. 
\par 

By construction, we require
\begin{equation}
    \begin{split}
        \ket{\psi(t = t_0)} = \hat{U}(t = t_0)\ket{\psi(t_0)} = \ket{\psi(t_0)}
        \\ \Rightarrow \hat{U}(t = t_0) = \ident.
    \end{split}
\end{equation}
Without loss of generality we can set $t_0 = 0$. 
Then, let us consider an infintesimally small time increment $\epsilon = t_0 + \Delta t$, such that $\epsilon \gg \epsilon^2$. 

We can say
\begin{equation}
    \hat{U}(\epsilon) = \ident + \mathcal{O}(\epsilon),
\end{equation}
which merely suggests that the time evolution operator
    at very small time is very close to the identity, with some small displacement proportional to the time. 
We suppose the form of the offset, so we can write
\begin{equation}
    \hat{U}(\epsilon) = \ident - \epsilon \left(\frac{i}{\hbar} \ho \right),
\end{equation}
    where the inclusion of the phase $-i$ is arbitrary, 
    and we have named as $\nicefrac{\ho}{\hbar}$ the operator by which the time evolution differs from the identity. 
In other words, the operator $\ho$ is generically the generator of the evolution/dynamics of $Q$;
    so far there is no restriction on $\ho$, 
    except that it must be of the same dimension as the Hilbert space in question. 
Recalling the unitarity condition, however:
\begin{equation}
    \label{eqn:hamiltonian_hermiticity}
    \begin{split}
        \hat{U}^{\dag}(\epsilon) \hat{U}(\epsilon) &= \ident
        \\ 
        \Rightarrow 
        \left( \ident + \frac{i}{\hbar} \epsilon \ho^\dag \right) \left( \ident - \frac{i}{\hbar} \epsilon \ho, \right)  &= \ident
        \\
        \Rightarrow
        \ident +  \frac{i}{\hbar} \epsilon (\ho^{\dag} - \ho) + \mathcal{O}(\epsilon^2) &= \ident
        \\
        \Rightarrow
        (\ho^{\dag} - \ho) = 0 
        \\
        \Rightarrow
        \ho^{\dag} = \ho.
    \end{split}
\end{equation}
\cref{eqn:hamiltonian_hermiticity} results in the condition for \emph{Hermiticity}, 
    meaning that $\ho$ is an observable of $Q$. 
In fact, this is the \emph{Hamiltonian} of the system, described in the next section. 

\par 
We can also use the infintesimal evolution to see 
\begin{equation}
    \label{eqn:derivation_schrodinger_eqn}
    \begin{split}
        \ket{\psi(t)} &= \hat{U}(t) \ket{\psi(t_0)}
        \\ \Rightarrow \ket{\psi(\epsilon)} &= \hat{U}(\epsilon) \ket{\psi(t_0)}
        \\ \Rightarrow 
        \ket{\psi(\epsilon)} &= \left(\ident - \epsilon \frac{i}{\hbar} \ho \right) \ket{\psi(t_0)}
        \\ \Rightarrow \ket{\psi(\epsilon)} &= \ket{\psi(t_0)} - \epsilon \frac{i}{\hbar} \ho \ket{\psi(t_0)}
        \\ \Rightarrow \frac{\ket{\psi(\epsilon)} - \ket{\psi(t_0)} }{\epsilon}  &= - \frac{i}{\hbar} \ho \ket{\psi(t_0)}
    \end{split}
\end{equation}
Taking the limit as $\epsilon \rightarrow 0 $, the left hand side of the final line of \cref{eqn:derivation_schrodinger_eqn} is the definition
    of the derivative of the wavefunction, $\frac{d \ket{\psi(t)}}{dt}$. 
Taken together, we have 
\begin{equation}
    \label{eqn:schrodinger}
    \frac{d}{dt} \ket{\psi(t)} = \frac{-i}{\hbar} \ho \ket{\psi(0)},
\end{equation}
    where $\ket{\psi(t)}$ is the wavefunction at time $t$, 
    $\ket{\psi(t_0)}$ is the wavefunction at $t$, such that $t > t_0$, 
    $\hbar = 1.054 \times 10{-34}$ is the reduced Planck constant and 
    $\ho$ is the \emph{Hamiltonian} of $Q$. 
For brevity we generally refer to $t_0 = 0$, and absorb $\hbar$ into $\ho$, which will later manifest in the Hamiltonian scalar parameters. 
\cref{eqn:schrodinger} is the most general form of \emph{\schrodinger equation}, 
    otherwise known as the \emph{time-dependent} \schrodinger equation; 
    we include it as \Cref{postulate:schrodinger_eqn} when describing the fundamentals of \gls{qm} (\cref{sec:postulates}), 
    since it can be seen as an irreducible equation of motion which is essential to the description of quantum systems. 

\par 


\subsection{Hamiltonians}\label{sec:hamiltonians}
In the previous section we introduced the Hamiltonian\footnotemark \ of $Q$ as the generator of its 
    time evolution dynamics;
    Hamiltonians are of primary importance in this thesis, so it is worth pausing to consider their physical meaning. 
We saw in \cref{eqn:hamiltonian_hermiticity} that $\ho$ is Hermitian, 
    meaning that the operator is physically observable 
    according to \Crefrange{postulate:eigenvalues}{postulate:hermiticity}
    of \glsentrylong{qm} (\cref{sec:postulates}). 
The Hamiltonian operator captures the energy of $Q$: 
    the eigenvalues of the observable $\ho$ are the permitted energy levels of the system.
\par 

The quantum Hamiltonian, $\ho$ is analogous to the classical Hamiltonian, 
    insofar as it captures all the interactions of a given system which contribute to its (dynamics).
Knowing the classical Hamiltonian and the initial conditions -- position and momentum -- 
    Hamilton's equations of motion allow for the calcaultion of those quantities for the particle 
    in question an infintesimal time later \cite{susskind2014classical}.    
Likewise, knowledge of the initial wavefunction, $\ket{\psi(t_0)}$, and the system's quantum Hamiltonian, $\ho$, 
    the quantum equations of motion, i.e. \schrodinger's equation \cref{eqn:schrodinger}, permits the calculation of the wavefunction
    at later times.
As such the Hamiltonian must consist of all processes which influence the evolution of $Q$;
    we will later break the Hamiltonian into independent \emph{terms} which each correspond to unique physical interactions
    $Q$ is subject to, \cref{sec:models}. 
We can think that each process/interaction $Q$ undergoes contributes to its total energy,
    giving intuition as to why its eigenvalues are the energy levels. 
\par 
\footnotetext{
    Aside: the author shares a hometown with the mathematician for whom it is named, William Rowan Hamilton. 
    It is hoped that, after another 150 years, the next physicist from Trim, Co. Meath, Ireland might 
    profitably use knowledge Hamiltonians on a quantum computer. 
}

Hamiltonians describe \emph{closed} quantum systems, 
    i.e. where \emph{all} processes and interactions which influence $Q$ are accounted for. 
Realistic quantum systems are influenced by a myriad of proximal systems, 
    and it is therefore infeasible to analytically account for them all. 
Instead, \emph{open} quantum systems' dynamics are described by Lindbladian operators, which encompass the Hamiltonian form. 
The Lindblad master equation is a generalisation of the \schrodinger equation, 
    providing the equation of motion for open quantum systems \cite{breuer2002theory, manzano2020short}.
In this thesis we only consider closed models for quantum systems;
    for meaningful impact of the techniques presented here, it will be necessary to expand them to account for the open system dynamics of realistic experiments.
We do, however, show initial progress towards this endeavour by modelling a physical system through a closed Hamiltonian, \cref{chapter:nv}.
\par 

\subsection{States and qubits}
The wavefunction is also known as its \emph{state}, which in general can be in superposition across its eigenstates, $\ket{i}$\footnotemark.  
The valid state space for $Q$ is its \emph{Hilbert space}, $\hilbert$,
    which is a generalisation of Euclidean vector space, 
    i.e. $\ket{\psi} \in \hilbert$. 
\footnotetext{We expand on this brief description in \cref{sec:states}, 
    and the Dirac notation emplyed throughout in \cref{sec:dirac_notation}.
}

\begin{subequations}
    \begin{equation}\label{eqn:state_vector}
        \ket{\psi} = \sum\limits_{i} \alpha_i \ket{i}
    \end{equation}
    \begin{equation}\label{vector_norm}
        \textrm{subject to} \ \ \ \sum\limits_{i} |\alpha_i|^2 =1, \ \ \ \alpha_x \in \mathbb{C}. 
    \end{equation}
\end{subequations}



For an ideal\footnotemark \ single particle, when the state, \cref{eqn:state_vector}, has two available eigenstates, 
    e.g. the horizontal ($H$) and vertical ($V$) polarisation of a single photon, 
    we can designate $Q$ as a two-level computational platform, called a \emph{qubit}, 
    analogous to the workhorse of classical computation, the bit. 
This is done by mapping each eigenstate to one of the orthogonal basis vectors,
    e.g. $\left{ \ket{H} = \ket{0} = \ket{V} = \icol{1 \\ 0}, \ket{1} = \icol{0 \\ 1}\right}$. 
\footnotetext{
    Here we restrict to the space of ideal, \emph{logical} qubits. In reality, physical qubits are beset by errors, 
    demanding error correction routines such that multiple particles are needed attain a single logical qubit. 
}

Then, a qubit's state can be written as 
\begin{equation}
    \label{eqn:qubit_state}
    \ket{\psi} = \alpha_0 \ket{0} + \alpha_1 \ket{1}. 
\end{equation}

Qubits can then be interfaced together 
\par 


\section{Quantum Computation and Simulation}
\begin{easylist}[itemize]
    & Algorithms
    && advantage
    & Hardware
\end{easylist}

Supremacy proposals: \cite{harrow2017quantum}
\par 

Bringing together the concepts of the chapter so far, 
    we can use qubits to represent the state of real quantum systems. 
This allows for efficient simuation of processes at the quantum level for the first time, 
    facilitiating novel insights of nature's most intricate details. 

