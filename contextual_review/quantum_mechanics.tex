The study of nature at the quantum level has been at the forefront of physics since the early $20^{th}$ century. 
\dots

Here we will only introduce concepts utilised in this thesis.
We elucidate some fundamental topics of linear algebra and quantum theory in \cref{apdx:fundamentals} for completeness, 
    but consider them too cumbersome to include in the main text. 
For a more complete and general introduction to \gls{qm}, the reader is referred to \cite{griffiths2018introduction}.

\section{Quantum Mechanics and Information}
\begin{easylist}[itemize]
    & states
    && qubits
    && pure/separable and mixed
    & operators/gates
    & Pauli matrices
    & Bloch sphere
    & measurement
    & expectation value
    & superposition/entanglement
    & Hamiltonian
    & Schrodinger equation
    & Unitary evolution
    & first/second quantisation
    & quantum tech other than computation/simulation
\end{easylist}

\section{Quantum Computation and Simulation}
\begin{easylist}[itemize]
    & Algorithms
    && advantage
    & Hardware
\end{easylist}

Supremacy proposals: \cite{harrow2017quantum}
\par 

Bringing together the concepts of the chapter so far, 
    we can use qubits to represent the state of real quantum systems. 
This allows for efficient simuation of processes at the quantum level for the first time, 
    facilitiating novel insights of nature's most intricate details. 

