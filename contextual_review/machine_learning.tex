\Gls{ml} is the application of statistics, algorithms and computing power to discover meaning and/or devise actions from data.
\gls{ml} has become an umbrella term, encompassing the family of algorithms
    which aim to leverage computaters to learn without being explicitly programmed,
    as opposed to the more general \gls{ai}, which seeks to make computers behave intelligently,
    admitting explicit programs to achieve tasks \cite{MLvAI}.
Its history is therefore imprecise since a number of early, apparently unrelated algorithms were proposed independently, 
    which now constitute \gls{ml} routines \cite{mcculloch1943logical, turing2009computing}. 
Nevertheless, the field of \gls{ml} has been advancing rapidly since the second half of the $20^{\textrm{th}}$ century \cite{russell2002artificial}, 
    especially recently due to the availability of advanced hardware such as \glspl{gpu}, 
    facilitating significant progress through an ever-increasing aresnal of powerful open source software \cite{pedregosa2011scikit, abadi2016tensorflow, paszke2019pytorch}. 
\par 

Throughout this thesis, we endeavour to combine known methods from the \gls{ml} literature with capabilities of \glspl{qc}\footnotemark. 
Typical \gls{ml} algorithms, which rely on \glspl{cpu} or \glspl{gpu}, are deemed \emph{classical} \glsentrylong{ml},
    in contrast with \gls{qml}, where \glspl{qc} are central to processing the data.
Similarly to the remit of \cref{chapter:qm}, here we do not provide an exhaustive account of \gls{ml} algorithms:
    rather, we describe only the algorithms which are used in later chapters, 
    referring readers to standard texts for a wider discussion \cite{russell2002artificial, hastie2009elements}

\footnotetext{Or simulated \glspl{qc}, in this thesis.}


\section{Classical \glsentrylong{ml}}\label{sec:classical_ml}
\begin{easylist}[itemize]
    & definition and aim(s)
    & supervised
    & unsupervised
    & example algorithms and applications
\end{easylist}


\section{Quantum \glsentrylong{ml}}
\begin{easylist}[itemize]
    & distinctions
    && q data q hardware $\rightarrow$ pure QML
    && q data classical hardware $\rightarrow$ ml for q physics
    && classical data q hardware $\rightarrow$ q enhanced ml
    && classical data c hardware $\rightarrow$ wrong thesis (\cref{sec:classical_ml})

    & examples/applications of QML 
    && QNN, q svm, 
    & Remit of this thesis $\rightarrow$ ml for q physics
    && i.e. using data from quantum system and/or hardware but in conjunction with classical co-processor, 
        for the study of quantum systems
\end{easylist}
\cite{dunjko2018machine}