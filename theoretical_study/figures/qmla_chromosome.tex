
\begin{table}

    \renewcommand{\arraystretch}{2.0}
    \setlength{\tabcolsep}{5pt}

    \def\rowbox#1#2{%
        \smash{\color{#2}\fboxrule=1pt\relax\fboxsep=2pt\relax%
        \llap{\rlap{\fbox{\vphantom{0}\makebox[#1]{}}}~}}\ignorespaces
    }

    \definecolor{parentA}{HTML}{002bd8} % blue
    \definecolor{parentB}{HTML}{316c00} % green
    \definecolor{mutation}{HTML}{961f29} % red
    \newcommand\parentcolourA{parentA}
    \newcommand\parentcolourB{parentB}
    \newcommand\mutationcolour{mutation}

    % \newcommand\parentcolourA{blue}
    % \newcommand\parentcolourB{green}
    % \newcommand\mutationcolour{red}

    \newcommand\longrowboxlenth{185pt}
    \newcommand\shortrowboxlength{80pt}
    
    \begin{center}
    \begin{tabular}{ c  c | c  c  c  c  c  c } 
            \hline
        \multicolumn{2}{c|}{Model} & \multicolumn{6}{c}{Chromosome} \\
        & $\terms$ & $\hat{\sigma}_{(1,2)}^{x}$ & $\hat{\sigma}_{(1,2)}^{z}$ & $\hat{\sigma}_{(2,3)}^{y}$ 
            & $\hat{\sigma}_{(2,3)}^{x}$ & $\hat{\sigma}_{(2,3)}^{y}$ & $\hat{\sigma}_{(2,3)}^{x}$ 
            \\ 
        \hline
        \textcolor{\parentcolourA}{$\gamma_{p_1}$} & $\irow{ 
            \textcolor{\parentcolourA}{ \hat{\sigma}_{(1,2)}^{x}} 
            & \textcolor{\parentcolourA}{ \hat{\sigma}_{(1,2)}^{z}} 
            & \textcolor{\parentcolourA}{ \hat{\sigma}_{(2,3)}^{y}} 
        }$
        & \rowbox{\longrowboxlenth}{\parentcolourA} 1 & 0 & 1 & 0 & 1 & 0 \\
        \textcolor{\parentcolourB}{$\gamma_{p_2}$} & $\irow{ 
            \textcolor{\parentcolourB}{ \hat{\sigma}_{(1,2)}^{z}} 
            & \textcolor{\parentcolourB}{ \hat{\sigma}_{(2, 3)}^{y}} 
            & \textcolor{\parentcolourB}{ \hat{\sigma}_{(2,3)}^{z}} 
        }$
        & \rowbox{\longrowboxlenth}{\parentcolourB} 0 & 0 & 1 & 0 & 1 & 1 \\

        \hline
        $\gamma_{c_1}$ & $\irow{ 
            \textcolor{\parentcolourA}{ \hat{\sigma}_{(1,2)}^{x}} 
            & \textcolor{\parentcolourA}{ \hat{\sigma}_{(1,2)}^{z}} 
            % & \textcolor{\parentcolourB}{ \hat{\sigma}_{(2, 3)}^{x}} 
            & \textcolor{\parentcolourB}{ \hat{\sigma}_{(2, 3)}^{y}} 
            & \textcolor{\parentcolourB}{ \hat{\sigma}_{(2,3)}^{z}} 
        }$
        & \rowbox{\shortrowboxlength}{\parentcolourA} 1 & 0 & 1 & \rowbox{\shortrowboxlength}{\parentcolourB} 0 & 1 & 1 \\
        $\gamma_{c_2}$ & $\irow{ 
            \textcolor{\parentcolourA}{ \hat{\sigma}_{(1,2)}^{z}} 
            & \textcolor{\parentcolourB}{ \hat{\sigma}_{(2, 3)}^{y}} 
        }$
        & \rowbox{\shortrowboxlength}{\parentcolourB} 0 & 0 & 1 & \rowbox{\shortrowboxlength}{\parentcolourA} 0 & 1 & 0\\

        \hline

        $\gamma_{c_2}^{\prime}$ & $\irow{ 
            \textcolor{\parentcolourA}{ \hat{\sigma}_{(1,2)}^{z}} 
            & \textcolor{\mutationcolour}{ \hat{\sigma}_{(2, 3)}^{x}} 
            & \textcolor{\parentcolourB}{ \hat{\sigma}_{(2, 3)}^{y}} 
        }$
        & 0 & 0 & 1 & \rowbox{10pt}{\mutationcolour} 1 & 1 & 0\\
        \hline 
    \end{tabular}

    \caption[Mapping between \gls{qmla}'s models and chromosomes used by a genetic algorithm.]{
        Mapping between \gls{qmla}'s models and chromosomes used by a genetic algorithm. 
        Example shown for a three-qubit system with six possible terms, $\s_{i,j}^{w} = \s_i^w \s_j^w$. 
        Model terms are mapped to binary genes: 
            if the gene registers $1$ ($0$) then the corresponding term is (not) present in the model.
        The top two chromosomes are \emph{parents}, $\gamma_{p_1}=101010$ (blue) and $\gamma_{p_2}=001011$ (green):
            they are mixed to spawn new models. 
        We use a one--point cross over about the midpoint:
            the first half of $\gamma_{p_1}$ is mixed with the second half of $\gamma_{p_2}$ 
            to produce two new offspring chromosomes, $\{\gamma_{c_1}, \gamma_{c_2}$\}. 
        Mutation occurs probabilistically: each gene has a 25$\%$ chance of being mutated, e.g. a single gene (red) flipping from $0 \rightarrow 1$ to mutate $\gamma_{c_2}$ to $\gamma_{c_2}^{\prime}$.
        The next generation of the genetic algorithm will then include $\{\gamma_{c_1}, \gamma_{c_2}^{\prime}\}$ (assuming $\gamma_{c_1}$ does not mutate). 
        To generate $N_m$ models for each generation, $N_m/2$ parent couples are sampled from the previous generation and crossed over. 
    }
    \label{table:chromosome_example}
    \end{center}
\end{table}
