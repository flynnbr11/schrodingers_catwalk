\begin{algorithm}
    \caption{\gls{es} subroutine: \ttt{generate\_models} via genetic algorithm}
    \label{alg:ga_generate_models}
    \DontPrintSemicolon
    \KwIn{ $\nu$ \tcp*[1]{information about models considered to date}}
    \KwIn{ $\tau$ \tcp*[1]{truncation rate}}\;
    \KwIn{ $g(\hi)$ \tcp*[1]{objective function that can act on any model $\hi$}}
    \KwIn{ \ttt{rank()} \tcp*[1]{function to rank models relative to each other}}
    \KwIn{ \ttt{truncate()} \tcp*[1]{function to truncate set of models}}
    \KwIn{ \ttt{normalise()} \tcp*[1]{function to normalise models' scores}}
    \KwIn{ \ttt{roulette()} \tcp*[1]{function to select models through roulette}}
    \KwIn{ \ttt{crossover()} \tcp*[1]{function to crossover two parents to produce offspring}}
    \KwIn{ \ttt{mutate()} \tcp*[1]{function to mutate offspring probabilistically}}\;

    \KwOut{$\mathbb{H}$ \tcp*[1]{set of models}}\;
    
    $ N_m = \lvert \nu \rvert $ \tcp*[1]{number of models}

    \For{$\hi \in \nu$}{
       $g_i \gets g(\hi)$ \tcp*[1]{model fitness via objective function}
    }\;
    $r \gets $ \ttt{rank($\{ g_i \}$)} \tcp*[1]{rank models by their fitness}
    $\mathbb{H}_t \gets $ \ttt{truncate($r, N_m\times \tau$)} \tcp*[1]{truncate models by rank: only keep $N_m \times \tau$}
    $ s \gets $ \ttt{normalise($\{g_i\}$)} $\forall \hi \in \mathbb{H}_t$ \tcp*[1]{normalise remaining models' fitnesses}

    $\mathbb{H} = \{\}$ \tcp*[1]{new batch of chromosomes/models}

    \While{ $\lvert \mathbb{H} \rvert < N_m$ }{
        $p_1, p_2 = $ \ttt{roulette($s$)} \tcp*[1]{use $s$ to select two parents via roulette selection}
    
        $c_1, c_2$ = \ttt{crossover($p_1, p_2$)} \tcp*[1]{produce offspring models}

        $c_1, c_2$ = \ttt{mutate($c_1, c_2$)} \tcp*[1]{probabilistically mutate}

        $\mathbb{H} \gets \mathbb{H} \cup \{ c_1, c_2 \} $ \tcp*[1]{add new models to batch}

    }\;

    % \ttt{roulette($s$)}

    return $\mathbb{H}$ 

\end{algorithm}
