In this Part, we examine a series of theoretical quantum systems, 
    to assess the usefulness of the \gls{qmla} protocol described in \cref{part:algorithms}. 
\par 
\vspace{1cm}

We begin in \cref{chapter:lattices} with ideal systems described by lattice models 
    under standard formalisms, i.e. Ising, Heisenberg and Hubbard models. 
This serves as a first test of the \gls{qmla} framework under reasonably straightforward
    conditions, where a small number of candidate models are proposed in advance, 
    with the \gls{true model} gauranteed to be among them. 
We then show that \gls{qmla} is capable also of classifying the family of 
    physical regimes to which a target model belongs. 
The initial idea for this chapter was proposed by Dr. Raffaele Santagati, 
    and refined together with Dr. Andreas Gentile and myself. 
I modified the \gls{qmla} software for this application, 
    ran the \glspl{instance} and analysed the data. 
The figures presented are my own.

\par 
\vspace{1cm}
In \cref{chapter:ga} we consider more general application of the \gls{qmla} protocol, 
    in significant model spaces comprising over 250,000 valid candidate models. 
We explore these spaces efficiently by incorporating a \glsentryfull{ga} in \gls{qmla}. 
I proposed \glspl{ga} for the study of large model spaces, 
    and performed the initial \gls{hyperparameter} tuning. 
I devised the numerous \glsentryfullpl{of} considered, 
    and in particular the combination of \glsentrylongpl{bf}
    with Elo ratings for a bespoke \gls{of} which takes advantage of \gls{qmla}'s core strengths. 
I built \gls{ga} infrastructure into the \gls{qmla} software, 
    ran the \glspl{instance} presented, and analysed the data.
The figures presented are my own. 
